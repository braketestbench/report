\section{Functional Requirements of the Testbench}\label{sec:functionalRequirements}

	According to the specification of the \textit{SAE J2522} and according to parameters considered to be important, some requirements for the testbench were defined.

	\begin{enumerate}
		\item Measure a braking pressure up to 16 MPa.
		\item Apply a braking pressure of at least 300 kPa.
		\item Measure temperatures up to 600 $^{\circ}$C
		\item Measure temperatures with a minimal resolution of 7.5 $^{\circ}$C.
		\item Accelerate the rotor to a speed up to 200 kph.
		\item System most have a sampling period of 50ms.
		\item The system hardware most be able to operate under temperatures up to 40$^{\circ}$C.
		\item System must two acquisition channels for temperature.
		\item System must have at least one channel for braking pressure acquisition.
		\item System must have a channel for vibration acquisition.
		\item System must have at least two digital outputs to control relays.
		\item System must have a channel for acquiring speed.
		\item The system must work with real time acquisition.
	\end{enumerate}
	
\section{Software Requirements}
	
	\begin{enumerate}
		\item System must have a sampling rate of 50 ms.
		\item System must be able to monitor six analog channels at once.
		\item System must be able to control the digital outputs and one analog output during acquisition without loosing the real time constrain.
		\item The data acquired does not need to be shown to user in real time.
		\item The software layer must be able to record the data of the test.
		\item The software highest layer must have a friendly GUI, advanced electronic and simple programmable knowledge cannot be a requirement to operate the software.
		\item Calibration of the sensors data must be easy to modify on the software.
		\item Software must be multiplatform.
	\end{enumerate}
