\chapter{Experimental Tests}\label{ch:experimentalTests}
		After the board assembly and fabrication phases were finished to ensure that this project attends its requirements. This chapter presents the experimental procedure, the results as well with the discussion and possible improovements. This chapter can be divided in two parts, Section \ref{sec:hardware-interfaces-test}, \textit{Hardware Interfaces Test}, will deal with the test of each acquisition channel and with the two analog outputs. Section \ref{sec:brake-test}, \textit{Brake Test}, deals with the tests with the materials from \textit{Laboratory of Wear of Materials}, this tests aim to validate the hardware solution as a whole.
		\par
		The experimental procedure is based on the brake test procedures proposed by \cite{saej2522} described in Section \ref{ssec:brake-tests}. Basically the test will compreheend turning the electric motor on (with the aid of the frequency inverter) and by monitoring the Speed Acquisition Channel (check Section \ref{sec:speed-acquisition-channel}) wait until the rotor frequency reaches a pre-defined upper limit and then after a configured delay time, stop accelerating the motor and starts to brake until the system reaches a lower speed limit. During the entire procedure the brake pressure, and the temperature on the disc shall be measured. This procedure should be repeated for a set number of times called snubs. 

		\section{Hardware Interfaces Test}\label{sec:hardware-interfaces-test}

	\subsection{Rotation Aquisition Channel Test}\label{sec:speed-aquisition-channel-test}
	
	\subsubsection{Configuration and Background}

		As mentioned in Section \ref{sssec:lm2907-basic-tachometer-circuit}, the \textit{Rotation Acquisition Channel} is based in a frequency-to-voltage converter. Hence, in order to test this acquisition channel it must be submitted to different input frequencies and this input frequency and the output voltage should be registered so that the circuit gain can be compaired to the theoretical value.
		\par
		The input of this circuit was wired to a function generator configured on the following way:

			\begin{itemize}
				\item High Voltage Level = 5V
				\item Low Voltage Level = -5V
				\item Waveform = Triangular
			\end{itemize}
		A triangular waveform was chosen because it is the most similar to a CKP signal as Figure \ref{fig:ckp-signal} shows. For the tests the input frequency was varried from nought to 150Hz (check Section \ref{sssec:lm2907-designed-circuit}) at a 10Hz step.
		
	\subsubsection{Results and Discussion}

		Table \ref{table:results-rotation-test} shows the results from the test described above.

			\begin{table}[h!]
				\centering
				\begin{tabular}{|l|l|l|l|}
					\hline
					\textbf{Frequency (Hz)} & \textbf{Output Read} & \textbf{Output Voltage (V)} & \textbf{Gain} \\ \hline
					0 & 170 & 0.831 & - \\ \hline
					10 & 198 & 0.968 & 13.67 \\ \hline
					20 & 226 & 1.105 & 13.68 \\ \hline
					30 & 254 & 1.241 & 13.68 \\ \hline
					40 & 282 & 1.378 & 13.68 \\ \hline
					50 & 310 & 1.515 & 13.68 \\ \hline
					60 & 338 & 1.652 & 13.68 \\ \hline
					70 & 366 & 1.789 & 13.68 \\ \hline
					80 & 394 & 1.926 & 13.68 \\ \hline
					90 & 422 & 2.063 & 13.68 \\ \hline
					100 & 450 & 2.199 & 13.68 \\ \hline
					110 & 478 & 2.336 & 13.68 \\ \hline
					120 & 506 & 2.473 & 13.68 \\ \hline
					130 & 534 & 2.610 & 13.68 \\ \hline
					140 & 561 & 2.742 & 13.65 \\ \hline
					150 & 589 & 2.879 & 13.65 \\ \hline
				\end{tabular}
				\caption{Rotation Acquisition Channel Test Results}
				\label{table:results-rotation-test}
			\end{table}

		The theoretical voltage output was calculate at Section \ref{sssec:lm2907-designed-circuit} and is described by Equation \ref{eqn:rotation-theoretical-vo} bellow.

			\begin{equation}\label{eqn:rotation-theoretical-vo}
				V_{O}=1.225V + 0.015 \cdot f_{IN}  \cdot \frac{V}{Hz}
			\end{equation}

		The average voltage gain from the circuit of 13.68Hz and the experimental error was 8.8$\%$, the standard deviation was only 0.01 Hz though. That means that even though the circuit gain is almost 10$\%$ different from the theoretical gain, the circuit gain varries very little along the frequency input range.
		\par
		The major difference the calculated to the real voltage output Equation is the voltage offset. Although it was designed to be 1.225V, the measured value was 0.831V, this is probably due to the lack of a high impedance buffer between the voltage reference circuit and the frequency-to-voltage converter offset pin. A future version of this project should consider an operational amplifier to fix this issue. However, as Table \ref{table:results-rotation-test} shows, the circuit has a linear output in respect to the input, the only difference during the tests is that test applications should consider a 0.831V offset instead of a 1.225V offset.
		\par
		The voltage output of the \textit{Rotation Acquisition Channel} can be calculated using Equation \ref{eqn:rotation-circuit-vo} bellow.
			
			\begin{equation}\label{eqn:rotation-circuit-vo}
				V_{O}=0.831V + 0.01368 \cdot f_{IN}  \cdot \frac{V}{Hz}
			\end{equation}
	\subsection{Temperature Acquisition Channels Test}\label{sec:temperature-acquisition-channels-test}
	
	\subsubsection{Configuration and Background}

		As mentioned in Section \ref{ssec:thermocouple-signal-conditioning}, the \textit{Temperature Acquisition Channel} as implemented using a IC that performs amplification, linearization and cold junction compensation, meaning it already gives a output voltage with a fixed gain of 5mv/$^{\circ}$C in respect to the measured temperature.
		\par
		In order to perform the test, the tip of the thermocouple connected to this circuit was placed touching the tip of another thermocouple connected to a multimeter from TEK instruments (model number ??). Both tips were heated using a soldering iron placed close to the thermocouple tips, the \textit{Temperature Acquisition Channel} was evaluating compairing the reads from the multimeter to the reads (0-1023) acquired through the acquisition serial port at every 20$^{\circ}$C interval.

	\subsubsection{Results}

		\subsubsubsection{Channel 1}

			Table \ref{table:results-temp1-test} shows the results from the test described above done at the first \textit{Temperature Acquisition Channel}.

				\begin{table}[h!]
					\begin{tabular}{|l|l|l|l|}
						\hline
						\textbf{Input Temperature (${\circ}$C)} & \textbf{Analog Read} & \textbf{Read Temperature (${\circ}$C)} \\ \hline
						40 & 43 & 42.03 \\ \hline
						60 & 65 & 63.54 \\ \hline
						80 & 83 & 81.13 \\ \hline
						100 & 105 & 102.64 \\ \hline
						120 & 124 & 121.21 \\ \hline
						140 & 146 & 142.72 \\ \hline
						160 & 170 & 166.18 \\ \hline
						180 & 191 & 186.71 \\ \hline
						200 & 210 & 205.28 \\ \hline
					\end{tabular}
					\caption{Temperature Acquisition Channel 1 Test Results}
					\label{table:results-temp1-test}
				\end{table}

			The average error from the circuit from the thermocouple tip to the value measured in the microcontroller's analog-to-digital-converter is 3.02$\%$ and the standard deviation is 1.25$\%$. 
			
		\subsubsubsection{Channel 2}

			Table \ref{table:results-temp2-test} shows the results from the test described above done at the second \textit{Temperature Acquisition Channel}.

				\begin{table}[h!]
					\begin{tabular}{|l|l|l|l|}
						\hline
						\textbf{Input Temperature (${\circ}$C)} & \textbf{Analog Read} & \textbf{Read Temperature (${\circ}$C)} \\ \hline
						40 & 42 & 41.06 \\ \hline
						60 & 63 & 61.58 \\ \hline
						80 & 83 & 80.16 \\ \hline
						100 & 103 & 101.66 \\ \hline
						120 & 125 & 122.19 \\ \hline
						140 & 147 & 143.70 \\ \hline
						160 & 170 & 167.16 \\ \hline
						180 & 192 & 187.68 \\ \hline
						200 & 206 & 201.37 \\ \hline
					\end{tabular}
					\caption{Temperature Acquisition Channel 2 Test Results}
					\label{table:results-temp2-test}
				\end{table}

			The average error from the circuit from the thermocouple tip to the value measured in the microcontroller's analog-to-digital-converter is 2.27$\%$ and the standard deviation is 1.06$\%$. 

	\subsubsection{Discussion}

		The used thermocouple amplifer datasheet \cite{ad8495-datasheet} specifies a maximum signal conditioning error of $\pm2\%$, considering that this test includes the error from the thermocouple tip to the serial port, only 1$\%$ higher for the first channel and only 0.27$\%$ higher for the second channel than the amplifier on error is a great result and shows the quality of the choosen solution.

	\section{Brake Force Acquisition Channels Test}\label{sec:brake-pressure-acquisition-channels-test}
	\section{Analog Speed Reference Output Channels Test}\label{sec:analog-speed-reference-output-channels-test}
		\section{Brake Test}\label{sec:brakeTest}

	\subsection{Full Stop Test}

	The desired test will be the full stop test, described by \cite{caixeta2017} as a test where the speed is set to a upper limit and then to nought rotation at the end. The test can also be carried with different braking pressure (different pressures at the hydraulic valve) in order to achieve more relevant data.

	\par
	In order to start the brake test the following parameters were configured on the \textit{Labview} Test Application:

	\begin{itemize}Number of Snubs: 
		\item\textit{Number of Snubs:} In order to achieve solid results, 50 snubs were set.
		\item\textit{Interval between snubs (s):} In order to let the brake pads cool, five seconds were set between each snub.
		\item\textit{Upper Limit (RPM):} The upper limit was set at 600 RPMs, that is almost the fastest measured rotation that the brake machine can achieve.
		\item\textit{Upper Wait Interval (s):} Three seconds, just enough so that the speed can stabilize.
		\item\textit{Lower Limit (RPM):} Npught, the idea is to fully stop the system.
		\item\textit{Lower Wait Interval (s):} Nought, as the \textit{Lower Limit (RPM)} will be zero, this time can be compreheended inside the \textit{Inverval Between Snubs}.
	\end{itemize}

	The test shall be carried twice, first with 1000kPa and later with 200kPa in order to compare results. The goal of this tests is not to evaluate braking performance of components of a brake system but rather to prove that the developed hardware can be used for that, i.e. showing that the measured quantities are directly related to the tests are being carried out. Hence, temperature should increase after each snub, the measured braking pressure should increase as the braking actuator is activated and the CKP signal frequency should start increasing shortly after the electric motor is turned on.

	\subsection{Results}

	Figure \ref{fig:1000k-braking-test-graph} shows the results for the test with 1000kPa...