\chapter{Experimental Tests}\label{ch:experimentalTests}
		After the board assembly and fabrication phases were finished to ensure that this project attends its requirements. This chapter presents the experimental procedure, the results as well with the discussion and possible improovements. This chapter can be divided in two parts, Section \ref{sec:channels-tests}, \textit{Hardware Interfaces Test}, will deal with the test of each acquisition channel and with the two analog outputs. Section \ref{sec:brake-test}, \textit{Brake Test}, deals with the tests with the materials from \textit{Laboratory of Wear of Materials}, this tests aim to validate the hardware solution as a whole.
		\par
		The experimental procedure is based on the brake test procedures proposed by \cite{saej2522} described in Section \ref{ssec:brake-tests}. Basically the test will compreheend turning the electric motor on (with the aid of the frequency inverter) and by monitoring the Speed Acquisition Channel (check Section \ref{sec:speed-acquisition-channel}) wait until the rotor frequency reaches a pre-defined upper limit and then after a configured delay time, stop accelerating the motor and starts to brake until the system reaches a lower speed limit. During the entire procedure the brake pressure, and the temperature on the disc shall be measured. This procedure should be repeated for a set number of times called snubs. 

		\section{Hardware Interfaces Test}\label{sec:hardware-interfaces-test}

	\section{Speed Aquisition Channel Test}\label{sec:speed-aquisition-channel-test}
	\subsection{Temperature Acquisition Channels Test}\label{sec:temperature-acquisition-channels-test}
	
	\subsubsection{Configuration and Background}

		As mentioned in Section \ref{ssec:thermocouple-signal-conditioning}, the \textit{Temperature Acquisition Channel} as implemented using an IC that performs amplification, linearization and cold junction compensation, meaning it already gives an output voltage with a fixed gain of 5mv/$^{\circ}$C in respect to the measured temperature.
		\par
		In order to perform the test, the tip of the thermocouple connected to this circuit was placed touching the tip of another thermocouple connected to a multimeter from TEK instruments (model number: TEK Instruments 10709). Both tips were heated using a soldering iron placed close to the thermocouple tips, the \textit{Temperature Acquisition Channel} was evaluating compairing the reads from the multimeter to the reads (0 - 1023) acquired through the acquisition serial port at every 20$^{\circ}$C interval.

	\subsubsection{Results}

		\subsubsubsection{Channel 1}

			Table \ref{table:results-temp1-test} shows the results from the test described above done at the first \textit{Temperature Acquisition Channel}.

				\begin{table}[h!]
				\centering
					\begin{tabular}{|l|l|l|l|}
						\hline
						\textbf{Input Temperature (${\circ}$C)} & \textbf{Analog Read} & \textbf{Read Temperature (${\circ}$C)} \\ \hline
						40 & 43 & 42.03 \\ \hline
						60 & 65 & 63.54 \\ \hline
						80 & 83 & 81.13 \\ \hline
						100 & 105 & 102.64 \\ \hline
						120 & 124 & 121.21 \\ \hline
						140 & 146 & 142.72 \\ \hline
						160 & 170 & 166.18 \\ \hline
						180 & 191 & 186.71 \\ \hline
						200 & 210 & 205.28 \\ \hline
					\end{tabular}
					\caption{Temperature Acquisition Channel 1 Test Results}
					\label{table:results-temp1-test}
				\end{table}

			The average error from the circuit from the thermocouple tip to the value measured in the microcontroller's analog-to-digital-converter is 3.02$\%$ and the standard deviation is 1.25$\%$. 
			
		\subsubsubsection{Channel 2}

			Table \ref{table:results-temp2-test} shows the results from the test described above done at the second \textit{Temperature Acquisition Channel}.

				\begin{table}[h!]
					\centering
					\begin{tabular}{|l|l|l|l|}
						\hline
						\textbf{Input Temperature (${\circ}$C)} & \textbf{Analog Read} & \textbf{Read Temperature (${\circ}$C)} \\ \hline
						40 & 42 & 41.06 \\ \hline
						60 & 63 & 61.58 \\ \hline
						80 & 83 & 80.16 \\ \hline
						100 & 103 & 101.66 \\ \hline
						120 & 125 & 122.19 \\ \hline
						140 & 147 & 143.70 \\ \hline
						160 & 170 & 167.16 \\ \hline
						180 & 192 & 187.68 \\ \hline
						200 & 206 & 201.37 \\ \hline
					\end{tabular}
					\caption{Temperature Acquisition Channel 2 Test Results}
					\label{table:results-temp2-test}
				\end{table}

			The average error from the circuit from the thermocouple tip to the value measured in the microcontroller's analog-to-digital-converter is 2.27$\%$ and the standard deviation is 1.06$\%$. 

	\subsubsection{Discussion}

		The used thermocouple amplifer datasheet \cite{ad8495-datasheet} specifies a maximum signal conditioning error of $\pm2\%$, considering that this test includes the error from the thermocouple tip to the serial port, only 1$\%$ higher for the first channel and only 0.27$\%$ higher for the second channel than the amplifier on error is a great result and shows the quality of the chosen solution.

	\subsection{Brake Force Acquisition Channels Test}\label{sec:brake-pressure-acquisition-channels-test}

	\subsubsection{Configuration and Background}

		As mentioned in Section \ref{ssec:load-cell-signal-conditioning}, the \textit{Brake Force Acquisition Channel} is a bridge amplifier with a gain of 500 and voltage output offset of 2.5V.
		\par
		In order to test this interface, the input of this channel will be submitted to input voltages from -4mv to 4mV at a 0.5mV step. This input voltage range was chosen because according the amplifier datasheet \cite{ina125}, when the device is powered up with 5V the voltage output can only typically go from 0.3V to 4.2V, and this input voltage range amplified by a factor of 500 with an offset of 2.5V will not go beyond the specified output voltage range.
		
	\subsubsection{Results}

		\subsubsubsection{Channel 1}

			Table \ref{table:results-load1-test} shows the results from the test described above done at the first \textit{Brake Force Acquisition Channel}.

				\begin{table}[h!]
				\centering
					\begin{tabular}{|l|l|l|l|}
						\hline
						\textbf{Input Voltage (mV)} & \textbf{Analog Read} & \textbf{Calculated Voltage (mV)} \\ \hline
						-4.0 & 97 &  474.10 \\ \hline
						-3.5 & 153 & 747.80 \\ \hline
						-3.0 & 205 & 1001.96 \\ \hline
						-2.5 & 250 & 1221.90 \\ \hline
						-2.0 & 360 & 1495.60 \\ \hline
						-1.5 & 357 & 1744.87 \\ \hline
						-1.0 & 401 & 1959.92 \\ \hline
						-0.5 & 438 & 2140.76 \\ \hline
						 0.0 & 495 & 2419.35 \\ \hline
						 0.5 & 555 & 2712.61 \\ \hline
						 1.0 & 615 & 3005.87 \\ \hline
						 1.5 & 632 & 3088.95 \\ \hline
						 2.0 & 697 & 3406.65 \\ \hline
						 2.5 & 728 & 3558.16 \\ \hline
						 3.0 & 815 & 3983.38 \\ \hline
						 3.5 & 855 & 4178.89 \\ \hline
						 4.0 & 921 & 4212.68 \\ \hline
					\end{tabular}
					\caption{Brake Force Acquisition Channel 1 Test Results}
					\label{table:results-load1-test}
				\end{table}

			The average experimental error for this amplification circuit was 2.43$\%$ and the standard deviation is 1.81$\%$. 

		\subsubsubsection{Channel 2}

			Table \ref{table:results-load2-test} shows the results from the test described above done at the second \textit{Brake Force Acquisition Channel}.

				\begin{table}[h!]
				\centering
					\begin{tabular}{|l|l|l|l|}
						\hline
						\textbf{Input Voltage (mV)} & \textbf{Analog Read} & \textbf{Calculated Voltage (mV)} \\ \hline
						-4.0 & 98 & 478.98 \\ \hline
						-3.5 & 154 & 752.69 \\ \hline
						-3.0 & 204 & 997.07 \\ \hline
						-2.5 & 243 & 1187.68 \\ \hline
						-2.0 & 304 & 1485.83 \\ \hline
						-1.5 & 347 & 1695.99 \\ \hline
						-1.0 & 404 & 1974.58 \\ \hline
						-0.5 & 462 & 2258.06 \\ \hline
						 0.0 & 506 & 2473.12 \\ \hline
						 0.5 & 551 & 2693.06 \\ \hline
						 1.0 & 600 & 2932.55 \\ \hline
						 1.5 & 669 & 3269.79 \\ \hline
						 2.0 & 701 & 3426.20 \\ \hline
						 2.5 & 766 & 3743.89 \\ \hline
						 3.0 & 822 & 4017.60 \\ \hline
						 3.5 & 852 & 4164.22 \\ \hline
						 4.0 & 993 & 4211.15 \\ \hline
					\end{tabular}
					\caption{Brake Force Acquisition Channel 2 Test Results}
					\label{table:results-load2-test}
				\end{table}

			The average experimental error for this amplification circuit was 1.92$\%$ and the standard deviation is 1.39$\%$. 
		

	\subsubsection{Discussion}

		With this test the \textit{Brake Force Acquisition Channel} amplifier showed satisfatory results, with errors of less than 3$\%$. The combination of this precise amplifier and the Precision Voltage Reference provided by this IC gives a great compound solution for dealing with load cell signals.
	\section{Analog Speed Reference Output Channels Test}\label{sec:analog-speed-reference-output-channels-test}
		\section{Brake Test}\label{sec:brakeTest}

	\subsection{Full Stop Test}

		The desired test will be the full stop test, described by \cite{caixeta2017} as a test where the speed is set to a upper limit and then to nought rotation at the end. In order to achieve more reliable data the test shall be carried a hundred times.

		\par
		In order to start the brake test the following parameters were configured on the \textit{Labview} Test Application:

		\begin{itemize}
			\item\textit{Number of Snubs:} In order to achieve solid results, 100 snubs were set.
			\item\textit{Interval between snubs (s):} In order to let the brake pads cool, five seconds were set between each snub.
			\item\textit{Upper Limit (RPM):} The upper limit was set at 500 RPMs, that is almost the fastest measured rotation that the brake machine can achieve.
			\item\textit{Upper Wait Interval (s):} Three seconds, just enough so that the speed can stabilize.
			\item\textit{Lower Limit (RPM):} Nought, the idea is to fully stop the system.
			\item\textit{Lower Wait Interval (s):} Nought, as the \textit{Lower Limit (RPM)} will be zero, this time can be compreheended inside the \textit{Inverval Between Snubs}.
		\end{itemize}

		The goal of this tests is not to evaluate braking performance of components of a brake system but rather to prove that the developed hardware can be used for that, i.e. showing that the measured quantities are directly related to the tests are being carried out. Hence, temperature should increase after each snub, the measured braking pressure should increase and the CKP signal frequency should decrease rapidly as the braking actuator is activated. Moreover, it shall be possible to see a variation in brake efficiency when the brakes are hotter.

	\subsection{Results and Discussion}

		Figure \ref{fig:test-first-ten-snubs} shows a graph with the measured quantities on the first ten of the hundred snubs.

		\begin{figure}[htbp]
				\centering
				\includegraphics[width=.8\textwidth]{figuras/fig-test-first-ten-snubs}
				\caption{Measurements of the first ten snubs of the test}
				\label{fig:test-first-ten-snubs}
		\end{figure}

		On the graph from Figure \ref{fig:test-first-ten-snubs} it is possible to see the correlation between the measured quantities. Clearly the measured rotation falls rapidly as the brake force saturates to a maximum brake force value. Moreover, at each brake event, detected when the brake force increases, the temperature measured on each of thoose thermocouples bounces.
		\par

		Taking a more detailed look at the temperature variation at each snub, Figure \ref{fig:test-first-ten-snubs-force-temperature} show the measured brake force regard the measured temperature at the first ten snubs. This graph states the coorelation between the temperature of the brake pads and the braking action, and the fact that the temperature increases during the braking act is perfectly in agreement with the fact that a braking system can be analysed as a kinectic to thermal energy converter, at is was explained in Section \ref{sec:working-principles-of-disk-brake-systems}.

		\begin{figure}[htbp]
				\centering
				\includegraphics[width=.8\textwidth]{figuras/fig-test-first-ten-snubs-force-temperature}
				\caption{Temperature and Brake Force Measurements of The First Ten Snubs}
				\label{fig:test-first-ten-snubs-force-temperature}
		\end{figure}

		Figure \ref{fig:test-first-ten-snubs-temperature} is similar to Figure \ref{fig:test-first-ten-snubs-force-temperature}, the difference is that this time only temperature measurement is shown in order to give a sharper view of how the temperature increases during each snub.

		\begin{figure}[htbp]
				\centering
				\includegraphics[width=.8\textwidth]{figuras/fig-test-first-ten-snubs-temperature}
				\caption{Temperature of The First Ten Snubs}
				\label{fig:test-first-ten-snubs-temperature}
		\end{figure}

		Figure \ref{fig:test-first-ten-snubs-force} is similar to the two previous figures, this time showing only the brake force along the first ten snubs.
		\begin{figure}[htbp]
				\centering
				\includegraphics[width=.8\textwidth]{figuras/fig-test-first-ten-snubs-force}
				\caption{Brake Force of The First Ten Snubs}
				\label{fig:test-first-ten-snubs-force}
		\end{figure}
		\par

		Whereas the temperature increases at each snub, there may be a point where the temperature reaches a equilibrium, Figure \ref{fig:test-temperature} shows the variation of temperature during the 100 snubs, it is possible to see that even that the temperature bounces at each snub, there may be a point were the temperature stabilizes close to a certain temperature.

		\begin{figure}[htbp]
				\centering
				\includegraphics[width=.8\textwidth]{figuras/fig-test-temperature}
				\caption{Temperature Variation During The Whole Test}
				\label{fig:test-temperature}
		\end{figure}
		\par

		Another interisting point to analyse is the effect of the temperature of the pads and the rate of desaceleration of the rotor. Figure \ref{fig:snubs-rotation} shows the desaceleration curves at each snub and shows that during the test the brake efficiency was slightly reduced as the temperature would increase.

		\begin{figure}[htbp]
				\centering
				\includegraphics[width=.8\textwidth]{figuras/fig-snubs-rotation}
				\caption{Deceleration During the Whole Test}
				\label{fig:snubs-rotation}
		\end{figure}