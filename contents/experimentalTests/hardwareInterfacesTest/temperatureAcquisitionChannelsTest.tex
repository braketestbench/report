\subsection{Temperature Acquisition Channels Test}\label{sec:temperature-acquisition-channels-test}
	
	\subsubsection{Configuration and Background}

		As mentioned in Section \ref{ssec:thermocouple-signal-conditioning}, the \textit{Temperature Acquisition Channel} as implemented using an IC that performs amplification, linearization and cold junction compensation, meaning it already gives an output voltage with a fixed gain of 5mv/$^{\circ}$C in respect to the measured temperature.
		\par
		In order to perform the test, the tip of the thermocouple connected to this circuit was placed touching the tip of another thermocouple connected to a multimeter from TEK instruments (model number: TEK Instruments 10709). Both tips were heated using a soldering iron placed close to the thermocouple tips, the \textit{Temperature Acquisition Channel} was evaluating compairing the reads from the multimeter to the reads (0 - 1023) acquired through the acquisition serial port at every 20$^{\circ}$C interval.

	\subsubsection{Results}

		\subsubsubsection{Channel 1}

			Table \ref{table:results-temp1-test} shows the results from the test described above done at the first \textit{Temperature Acquisition Channel}.

				\begin{table}[h!]
				\centering
					\begin{tabular}{|l|l|l|l|}
						\hline
						\textbf{Input Temperature (${\circ}$C)} & \textbf{Analog Read} & \textbf{Read Temperature (${\circ}$C)} \\ \hline
						40 & 43 & 42.03 \\ \hline
						60 & 65 & 63.54 \\ \hline
						80 & 83 & 81.13 \\ \hline
						100 & 105 & 102.64 \\ \hline
						120 & 124 & 121.21 \\ \hline
						140 & 146 & 142.72 \\ \hline
						160 & 170 & 166.18 \\ \hline
						180 & 191 & 186.71 \\ \hline
						200 & 210 & 205.28 \\ \hline
					\end{tabular}
					\caption{Temperature Acquisition Channel 1 Test Results}
					\label{table:results-temp1-test}
				\end{table}

			The average error from the circuit from the thermocouple tip to the value measured in the microcontroller's analog-to-digital-converter is 3.02$\%$ and the standard deviation is 1.25$\%$. 
			
		\subsubsubsection{Channel 2}

			Table \ref{table:results-temp2-test} shows the results from the test described above done at the second \textit{Temperature Acquisition Channel}.

				\begin{table}[h!]
					\centering
					\begin{tabular}{|l|l|l|l|}
						\hline
						\textbf{Input Temperature (${\circ}$C)} & \textbf{Analog Read} & \textbf{Read Temperature (${\circ}$C)} \\ \hline
						40 & 42 & 41.06 \\ \hline
						60 & 63 & 61.58 \\ \hline
						80 & 83 & 80.16 \\ \hline
						100 & 103 & 101.66 \\ \hline
						120 & 125 & 122.19 \\ \hline
						140 & 147 & 143.70 \\ \hline
						160 & 170 & 167.16 \\ \hline
						180 & 192 & 187.68 \\ \hline
						200 & 206 & 201.37 \\ \hline
					\end{tabular}
					\caption{Temperature Acquisition Channel 2 Test Results}
					\label{table:results-temp2-test}
				\end{table}

			The average error from the circuit from the thermocouple tip to the value measured in the microcontroller's analog-to-digital-converter is 2.27$\%$ and the standard deviation is 1.06$\%$. 

	\subsubsection{Discussion}

		The used thermocouple amplifer datasheet \cite{ad8495-datasheet} specifies a maximum signal conditioning error of $\pm2\%$, considering that this test includes the error from the thermocouple tip to the serial port, only 1$\%$ higher for the first channel and only 0.27$\%$ higher for the second channel than the amplifier on error is a great result and shows the quality of the chosen solution.
