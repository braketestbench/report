\subsection{Full Stop Test}\label{ssec:full-stop-test}

		The desired test will be the full stop test, described by \cite{caixeta2017} as a test where the speed is set to an upper limit and then to nought rotation at the end. In order to evaluate the reliability from developed hardware.the test shall be carried a hundred times.

		\par
		In order to start the brake test the following parameters were configured on the \textit{Labview} Test Application as Table \ref{table:brake-test-parameters} shows.

		\begin{table}[h!]
			\centering
			\begin{tabular}{|l|l|l|}
				\hline
				\textbf{Parameter} & \textbf{Value} & \textbf{Description} \\ \hline
				Number of Snubs & 100 & To achieve reliable data. \\ \hline
				Interval Between Snubs & 5s & Allows the brakes to cool down after each Snub. \\ \hline
				Upper Limit & 500 rpm & Almost the maximum rotation of the rotor (580rpm) \\ \hline
				Upper Wait Interval & 3s & Just enough so that the speed can stabilize \\ \hline
				Lower Limit & 0 rpm & The ideia is to fully stop the system. \\ \hline
				Lower Wait Interval & 1s & Just enough to guarantee no momentum is present. \\ \hline
			\end{tabular}
			\caption{Test Parameters Setup}
			\label{table:brake-test-parameters}
		\end{table}

		The goal of this tests is not to evaluate braking performance of components of a brake system but rather to achieves that the developed hardware can be used for that, i.e. showing that the measured quantities are directly related to the tests are being carried out. Hence, it is expected that temperature should increase after each snub. Morover, as the measured braking pressure increases and the CKP signal frequency should decrease rapidly. Moreover, it shall be possible to see a variation in brake efficiency when the brakes starts to get hotter.