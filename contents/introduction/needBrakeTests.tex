\section{The need for performing brake tests}
		The brake system is a critical part of an automobile, thanks to this system it is possible to use the latter under safe conditions both in urban and rural areas. There are some ideal requirements that a brake system should be able to attend \cite{kawaguchi} :

		\begin{itemize}
			\item Reduce the speed of a moving vehicle, increasing the deceleration of the same.
			\item Stop the vehicle completely.
			\item Maintain the vehicle speed, preventing unwanted acceleration in downhill paths.
			\item Keep the vehicle motionless while it is parked.
		\end{itemize}

		It is important to emphasize that this conditions are ideal, considering that in extremely hazardous or stressful situations the system might not operate properly and will not attend thoose previous requirementes. Considering the importance of brake system the same need to have minimal breaking capacity so vehicles can be decelerated with greater efficience. 
		\par
		In contrast, more effective brake systems means more cost to manufactures and consenquently to customers. Theoretically this would meant that manufactures need to choose a trade-off between quality and cost. However, the point in which this trade-off is setted is determined by governament regulations. Moreover if there was no general regulations each car manufecturer would have a standard that they judge is sufficient. In Brazil the governament partitions that define this regulations are the \textit{National Traffic Council} and the \textit{National Institute of Meteorology, Quality and Technology}, most of those regulations are based in the european regulation ECE-13/05 \cite{inmetro2013} .
		\par
		Considering the importance of regulatory standards, the need for brake tests becomes even more evident as it is mandatory to ensure that brake-systems will attend to regulations requirementes. Only with extensive testing it is possible to ensure that a particular system will attend to all standards regarding it's category of operation. 
		\par
		Making all theese considerations, a \textit{Break-System-Testbench} may be considered a useful device for the automotive industry. Considering that it would be able to simulate a close enough replica of real evironments and situations that a brake system is submitted, this testbench could allow car manufactures and break system parts manufactures to avoid expenses in tests as they would be able to test different parts of the system in a assisted and controlled evironment.
