		The brake system is a critical part of an automobile, thanks to this system it is possible to use the latter under safe conditions both in urban and rural areas. There are some ideal requirements that a brake system should be able to attend \cite{kawaguchi}:

		\begin{itemize}
			\item Reduce the speed of a moving vehicle, increasing the deceleration of the same.
			\item Stop the vehicle completely.
			\item Maintain the vehicle speed, preventing unwanted acceleration in downhill paths.
			\item Keep the vehicle motionless while it is parked.
		\end{itemize}

		\par
		Another point of view should also be considered during this analysis, the manufacturers point of view. More effective brake systems means more research and probably more expensive materials, generating more cost to manufacturers and consequently more cost to customers. Theoretically this would meant that manufacturers need to choose a trade-off between quality and cost. However, the point in which this trade-off is setted is determined by government regulations. Moreover if there was no general regulations each car manufecturer would have a standard that they judge is sufficient. In Brazil, according to Resolution N.519 \cite{contran519} from the National Traffic Council \textit{(CONTRAN - Conselho Nacional de Trânsito)}, the minimal requirements of perfomance of brake systems from any vehicle with 750kg or less should meet the following regulations from Brazilian Association of Technical Standards \textit{(ABNT - Associação Brasileira de Normas Técnicas)}: \textit{NBR 10966-1}, \textit{NBR 10966-2}, \textit{NBR 10966-3}, \textit{NBR 10966-4}, \textit{NBR 10966-5}, \textit{NBR 10966-6}, \textit{NBR 10966-7} and \textit{NBR 16068}. Most of these regulations are based in the european regulation ECE-13/05.
		\par
		Considering the importance of regulatory standards, the need for brake tests becomes even more evident as it is mandatory to ensure that brake-systems will attend to regulations requirementes. Only with extensive testing it is possible to ensure that a particular system will attend to all standards regarding it's category of operation. 
		\par
		Making all these considerations, a \textit{Break-Testbench} can be a useful device for the automotive industry and research. Considering that this device would be able to simulate a replica of real evironments and situations that a brake system is submitted, this testbench could allow manufactures and suppliers to avoid expenses in full scale tests as they would be able to test different parts of the system in an assisted and controlled evironments.