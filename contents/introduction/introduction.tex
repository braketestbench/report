With the advancement of technology cars are leaving the factory each time with more power for more affordable prices. In 1995 the best-selling car in Brazil \cite{lideres-vendas-brasil} \textit{(Volkswagen Gol Plus 1.0)} had 49.8hp of maximum power and brand-new would accelerate from nought to 100kph in 22.4 seconds \cite{cnwgol1993} while sales champion of 2015 \textit{Volkswagen Gol 1.0} had 76hp of maximum power and could do the same challenge in 13.3 seconds \cite{cnwgol2013}, almost half the time from the previous. Interisting fact is even though the latter is 20 years younger, both cars have similar brake systems, disk brakes in the front and drum ones in the back. To make things worse, the younger one does not have ABS \textit{Anti-lock Braking System}, which only became mandatory for cars manufactured from 2014 and beyond according to brazilian regulations.
	\par
	Although this is short analysis has only two subjects it brings up that maybe manufactures and customer are too focused in performance rather than safety. Of course for a standard customer it is obviously hard too evaluate the breaking performance of a vehicle upon buying it. Governament and legal authorities have been creating strict regulations for manufactures too follow in order to ensure that cars have a higher standard of safety.
	\par
	Brake systems are extremely important in terms of safety because even though cars nowadays are required to have a higher perfomances in crash tests it is always favored too avoid collisions.
	\par
	Brake tests with full scale vehicles are expensive and this somehow makes extensive testing unfeasible. Also the time required for each test might be a constrain. It is a possibility that maybe using small scale tests it would still be possible to provide relevant information about quality and performance of brake systems with lower costs and reduced time. Small scale tests do not have the purpose of fully replacing full scale ones, but the savings in costs and time that they can provide could be used for mass testing, and this can already show their utility and relevance \cite{gardinalli2005comparaccao}. 
	\par
	Judging the brake efficience of a vehicle as a whole involves a lot of factors, a small scale test will not provide results that could be used directly to address the quality of a car break system but it is possible to focus the results in the performance of individual componentes of the system such as pads, disks and calipers \cite{halderman2016automotive}, and evaluating the performance of this components is a good start for judge the brake capacity of a brake system.
	\par
	Braking tests have been carried out for years and have been regulated for some time. A international standard for brake testing has been the regulation \textit{SAE J2522} \cite{sae}, it gives a the description of how break tests should be conducted for evaluating low weight passengers cars.
