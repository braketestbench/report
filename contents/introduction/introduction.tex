	With the advancement of technology, cars are leaving the factory each time with more power for more affordable prices. In 1995 the best-selling car in Brazil \cite{lideres-vendas-brasil}, \textit{(Volkswagen Gol Plus 1.0)}, had 49.8hp of maximum power and brand-new would accelerate from nought to 100kph in 22.4 seconds \cite{cnwgol1993}. On the other hand, the sales champion of 2015 ,\textit{Volkswagen Gol 1.0}, had 76hp of maximum power and could do the same task in 13.3 seconds \cite{cnwgol2013}, almost half the time from the previous. Interisting fact is that even though the latter is 20 years younger, both cars have similar brake systems: disk brakes in the front wheels and drum ones in the back. This enhancement on vehicle accelerations naturally imply on higher top speeds, which should lead to a bigger concern in brake systems effectiveness.
	\par
	During the development of any solution or enhancement on any technology or product, testing is fundamental \cite{taguchi1986introduction}. In the matter of brake systems, the fact stated on the previous sentence can also be considered true. On the process of developing more effective brake systems, it is important to perform brake tests.
	\par
	Although brake tests are so importatant, performing this tests with full scale vehicles are obviously expensive and this somehow makes extensive testing unfeasible. Also the time required for each test might be a constrain. It is a possibility that maybe using a small scale test plataform would still make it possible to provide relevant and reliant information about quality and performance of brake systems with lower costs and reduced testing time. Small scale tests do not have the purpose of fully replacing full scale ones, but the savings in costs and time that they could provide can be used for mass testing, and this can already somehow show their utility and relevance \cite{gardinalli2005comparaccao}. 
	\par
	Evaluating the brake efficiency of a vehicle as a whole, involves a lot of factors, a small scale test plataform will not provide results that could be used to fully address the quality of a car break system, instead, it is possible to focus the results on the performance of individual componentes of the system such as pads, disks and calipers \cite{halderman2016automotive}, and evaluating the performance of this components is a good start for evaluating the braking capacity of a brake system.
	\par
	Governament and legal authorities have been creating strict regulations for manufactures to follow in order to ensure that cars have a higher standard of safety. Braking tests have been regulated for some time. A international standard for brake testing is the regulation \textit{SAE J2522} \cite{sae}, it gives a the description of how break tests should be conducted for systems used in low weight passengers cars.
