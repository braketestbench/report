\section{Voltage Supplies}\label{sec:voltage-supplies}

	\subsection{Power Supplies}\label{ssec:power-supplies}
		In order to avoid any interference of the AC line, this project will work with a DC voltage input and any other necessary voltages will be acquainted from this higher input voltage supply. There are many different types of voltage regulators, nowadays the most common DC/DC being switching regulators. They are more efficient than linear regulators and consequently they dissipate less heat, the downside is that they tend to have some ripple at the output \cite{schweber2017}. However, as the ripple can easily be filtered and that a switching power supply allows the input voltage to be of a much bigger range, this solution was considered the best for the project. 

		\subsubsection{5V Supply}\label{sssec:5v-supply}
			The voltage supply was chosen to be a 5V DC supply because all the components can work with this voltage. The chosen voltage regulator for the 5V supply was the TL2575-05 from \textit{Texas Instruments} \cite{tl2575-05-datasheet}, it has an output up to 1A, voltage drop of 2V, typical efficiency of 88$\%$ and can work with a 7 to 42V input. Figure \ref{fig:tl2575-05-circuit} shows the circuit used for the 5V supply.

			\begin{figure}[htbp]
				\centering
					\includegraphics[width=1\textwidth]{figuras/fig-tl2575-05-circuit.png}
				\caption{5V Power Supply Circuit}
				\label{fig:tl2575-05-circuit}
			\end{figure}

			According to \cite{tl2575-05-datasheet}, the main components to be chosed for the switching regulator are L4, C38 and D9. The inductor and the capacitor are used to form a LPF to transform the PWM output of the switching regulator to a DC voltage. D9 is a flyback diode used to shunt any negative current from the switching to ground and protect the device. The datasheet recommends using a 3A reverse current diode, a 330uH inductor and 330uF capacitor for the 5V regulator.

			The other components of the circuit are:

			\begin{itemize}
				\item\textit{TZ10}: TVS diode to clamp to protect the input from ESD.
				\item\textit{F1}: resetable fuse, they will limit the current that flows into the circuit and protect the TVS.
				\item\textit{D8}: A diode to protect the input from inverse polarity.
				\item\textit{C37}: Input bypass capacitor recommended by the device's datasheet.
				\item\textit{C39}: Another bypass capacitor on the output to filter any residual noise.
			\end{itemize}

	\subsection{Voltage References}\label{ssec:voltage-references}

		In this project any power supply that the precision and stability of the output voltage is more concerning than the maximum output current will be called a voltage reference.

		\subsubsection{4V1 Reference}\label{sssec:4v1-reference}

			As said in Section \ref{ssec:thermocouple-sensor-detection}, the sensor detection circuit needs a 4V1 voltage reference for the comparator. This is achieved using the LM4040-4.1 from \textit{Microchip} \cite{lm4040-datasheet}. This is a fixed 4V1 precise voltage reference and according to the datasheet it has a tolerance of $\pm1.15\%$. Figure \ref{fig:4v1-voltage-ref} shows the circuit for the 4V1 voltage reference.

			\begin{figure}[htbp]
				\centering
					\includegraphics[width=.5\textwidth]{figuras/fig-4v1-voltage-ref}
				\caption{4V1 voltage reference circuit}
				\label{fig:4v1-voltage-ref}
			\end{figure}

			According to the datasheet the current flowing through the LM4040-4.1 must be never be greater than 15mA, voltage reference pins do need very little current, so a third of this 15mA will be taken as goal. Considering the voltage drop of 0.9V (5V - 4.1V) and the current of 5mA we need a 180$\Omega$ resistor in series with the regulator.