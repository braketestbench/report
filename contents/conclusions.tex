\chapter{Conclusions and Future Endavours}\label{ch:conclusions-and-future-endavours}
		\section{Conclusions}\label{sec:conclusions}
The goals of this work can be summarized as being the project of an electronic test bench to perform brake tests in automotive components, and this goal was achieved.
		\par
		As the tests proved the testbench can perform the required data acquisition of the specified physical quantities listed on Section \ref{sec:monitored-parameters} and also meet the mandatory requirements from Section \ref{sec:functionalRequirements} as well as the system can control the actuators needed for the brake tests.
		\par
		During the project phase of this system, versatility was always in mind, the system does not only have the basical functionalities specified on \textit{SAEJ2522}, it also achieved all the optional requirements from Section \ref{sec:functionalRequirements}. At the end, a multifunctional hardware with transient protection in all ports was delivered. An interesting point of the developed hardware is that as it operates with pre-defined commands comming from its generic USB emulated serial port, any higher level software can theoretically be used to control it, as long as it respects the defined communication protocol.
		\par
		The results of the brake tests performed shows a crucial point of this system, repeatability, the braketests were carried out for a hundred snubs without any interruption or bug, making the system extremely reliable. The transient protections circuits worked really well, even in the noisy and transient-dangerous environment created by the three-phase motor and frequency-inverter, not a single circuit was damaged during the tests.
		\par
		As mentioned before, this system is extremely versatile, making it able to perform many different functions. Although this system was initially developed with the final goal of performing braketests, at another point of view it can be seen as a control and acquisition system. For example, considering it has thermocouple reading inputs and digital inputs, it could be used to control a SMT (Surface Mount Technology)  reflow oven temperature. Moreover the force acquistion channels can be used to control an industrial precision scale. Considering a vibration channel was implemented, even a vibration testbench could be controlled and monitored. Possibilities are infinite, the only restriction is the amount of inputs and outputs of the designed hardware.

	\section{Future Endavours}\label{sec:future-endavours}
		A possible interesting improvement for this project would be developing a personalized software instead of using a default \textit{LabView} application. An open source plataform could make the system more capable of being readily changed and maybe developing a system using Java could even make it easier to migrate the system to a web plataform.
		\par
		A future version of this project could maybe consider switching from an integrated architecture to a distribuited one, instead of putting all acquisition channels in one board, a single board for each acquisition channel could be created and all this boards could communicate through a I2C bus for example. That way the system versatility would be only limit to the number of allowed devices on the bus. Moreover, as the sensors cables could be smaller (peripheral boards would be placed closer to the sensors), there would probably be less noise captured by those cables.

